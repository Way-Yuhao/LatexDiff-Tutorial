%%%%%%%%%%%%%%%%%%%%%%%%%%%%%%%%%%%%%%%%%%%%%%%%%%%%%%%%%%%%%%%
%
% Welcome to Overleaf --- just edit your LaTeX on the left,
% and we'll compile it for you on the right. If you open the
% 'Share' menu, you can invite other users to edit at the same
% time. See www.overleaf.com/learn for more info. Enjoy!
%
%%%%%%%%%%%%%%%%%%%%%%%%%%%%%%%%%%%%%%%%%%%%%%%%%%%%%%%%%%%%%%%
\documentclass{article}
\title{latexdiff}
\author{me\and you}
\date{May 2023}

\begin{document}

\maketitle

\section{Introduction}
Timely and accurate detection of hurricane debris is critical for effective disaster response and community resilience. While post-disaster aerial imagery is readily available, robust debris segmentation solutions applicable across multiple disaster regions remain limited.
% Developing a generalized solution is challenging due to varying environmental and imaging conditions that alter debris' visual signatures across different regions, further compounded by the scarcity of training data.
Since environmental and imaging conditions vary by region and alter debris’ visual signatures, developing a debris segmentation model suitable for different hazard scenarios is challenging, especially given the scarcity of training data~\cite{wu2020phrasecut}.
% This study addresses these challenges by fine-tuning pre-trained foundation vision models, achieving robust performance with a relatively small, high-quality dataset.
This study addresses these challenges by relying on vast pretrained knowledge from the foundation vision model CLIP and fine-tuning a lightweight decoder on labeled debris imagery, thereby achieving robust performance in segmenting non-vegetative debris within complex urban backgrounds with a relatively small, high-quality dataset.
To enable fine-tuning, this work introduces an open-source dataset comprising approximately 1,200 manually annotated aerial RGB images of structural debris from Hurricanes Ian, Ida, and Ike.
To mitigate human biases and enhance data quality, labels from multiple annotators are strategically aggregated into consensus annotations.
% and visual prompt engineering is employed.
The resulting fine-tuned model, named \emph{CLIPSeg-debris}, achieves a Dice score of 0.86 on data from Hurricane Ida---a disaster event entirely excluded during training---outperforming baseline models.
% with virtually no false positives in debris-free areas.
% This work presents the first event-agnostic debris segmentation model requiring only standard RGB imagery during deployment, making it well-suited for rapid, large-scale post-disaster impact assessments and recovery planning.
This approach thereby enables broader adoption across multiple hazard scenarios.
effectively adapts to various backgrounds, sensor resolutions, and debris conditions.

\bibliographystyle{apalike}
\bibliography{additional_ref}
\end{document}